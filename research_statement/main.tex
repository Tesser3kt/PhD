\documentclass[a4paper,oneside,svgnames]{amsart}

\usepackage[T1]{fontenc}
\usepackage[utf8]{inputenc}

\usepackage{xcolor}
\usepackage[colorlinks=true,citecolor=Green]{hyperref}
\usepackage{amsrefs}

\title{Research Statement}
\author{Adam Klepáč}
\date{\today}

\begin{document}
 \maketitle
 \begin{abstract}
  Following the construction of $d$-representation-finite algebras
  in~\cite{iyama} and the description of the correspondence between certain
  types of cluster algebras and triangulations of bordered surfaces with marked
  points in~\cite{fst}, links have appeared connecting $d$-representation finite
  algebras to higher dimensional variants of said surface. One such link was
  discovered in~\cite{ot} between higher Auslander algebras of the path algebra
  of linearly oriented Dynkin quiver $A_n$ and cyclic polytopes. I wish to
  further study such kinds of connections, starting with the establishment of a
  similar type of link for path algebras of quivers of type $D_n$ which, in the
  low-dimensional case, correspond to once punctured polygons; then, with a
  touch of expectation and naïvety, broadening it to include (special types of)
  cluster algebras not necessarily representation-finite.
 \end{abstract}

 \section{Introduction}
 \label{sec:introduction}

 This text serves primarily as an overview of relevant concepts regarding
 cluster algebras, bordered surfaces with marked points, higher dimensional
 cluster categories and $d$-representation-finite algebras interwoven with ideas
 of possible generalizations and caveats tied to such endeavour. So far, I have
 only scratched the surface of this topic, hence very few original results are
 present.

 % TODO links
 In Section 2, I give a summary of the theory of bordered surfaces with marked
 points. Section 3 is dedicated to (normalized skew-symmetrizable) cluster
 algebras and their connection to bordered surfaces with marked points is drawn.
 Sections 4 and 5 define $d$-representation-finite algebras and higher cluster
 categories, respectively. Section 6 summarizes relevant results
 from~\cite{ot}, regarding a higher-dimensional kind of connection described in
 Section 3. Finally, Section 7 is riddled with (splinters of) steps towards
 generalizations of the content of Section 6.

 \bibliography{refs.bib}
\end{document}
