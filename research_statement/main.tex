\documentclass[a4paper,oneside,svgnames,draft]{amsart}

\usepackage[T1]{fontenc}
\usepackage[utf8]{inputenc}

\usepackage{xcolor}
\usepackage[colorlinks=true,citecolor=Green]{hyperref}
\usepackage{amsrefs}
\usepackage{tikz}
\usetikzlibrary{arrows,math}
\usepackage[noabbrev]{cleveref}
\usepackage{caption}
\usepackage{subcaption}
\usepackage{mathtools}

% Theorems
\theoremstyle{plain}
\newtheorem{theorem}{Theorem}[section]
\newtheorem{proposition}[theorem]{Proposition}
\newtheorem{lemma}[theorem]{Lemma}

\theoremstyle{definition}
\newtheorem{definition}[theorem]{Definition}

\let\hom\relax
\let\mod\relax
\DeclareMathOperator{\ext}{Ext}
\DeclareMathOperator{\add}{add}
\DeclareMathOperator{\hom}{Hom}
\DeclareMathOperator{\mod}{mod}
\DeclareMathOperator{\gldim}{gl.dim}

\title{Research Statement}
\author{Adam Klepáč}
\date{\today}

\begin{document}
 \maketitle
 \begin{abstract}
  Following the construction of $d$-representation-finite algebras
  in~\cite{iyama} and the description of the correspondence between certain
  types of cluster algebras and triangulations of bordered surfaces with marked
  points in~\cite{fst}, links have appeared connecting $d$-representation finite
  algebras to higher dimensional variants of said surface. One such link was
  discovered in~\cite{ot} between higher Auslander algebras of the path algebra
  of linearly oriented Dynkin quiver $A_n$ and cyclic polytopes. I wish to
  further study such kinds of connections, starting with the establishment of a
  similar type of link for path algebras of quivers of type $D_n$ which, in the
  low-dimensional case, correspond to once punctured polygons; then, with a
  touch of expectation and naïvety, broadening it to include (special types of)
  cluster algebras of finite mutation type.
 \end{abstract}

 \section{Introduction}
 \label{sec:introduction}

 This text serves primarily as an overview of relevant concepts regarding
 cluster algebras, bordered surfaces with marked points, higher dimensional
 cluster categories and $d$-representation-finite algebras interwoven with ideas
 of possible generalizations and caveats tied to such endeavour. So far, I have
 only scratched the surface of this topic, hence very few original results are
 present.

 % TODO links
 In Section 2, I give a summary of the theory of bordered surfaces with marked
 points. Section 3 is dedicated to (normalized skew-symmetrizable) cluster
 algebras and their connection to bordered surfaces with marked points is drawn.
 Sections 4 and 5 define $d$-representation-finite algebras and higher cluster
 categories, respectively. Section 6 summarizes relevant results
 from~\cite{ot}, regarding a higher-dimensional kind of connection described in
 Section 3. Finally, Section 7 is riddled with (splinters of) steps towards
 generalizations of the content of Section 6.

 \section{Bordered Surfaces with Marked Points}
 \label{sec:bordered-surfaces-with-marked-points}

 This section is a brief summary of~\cite{fst}, Section 2.
 \begin{definition}[Bordered surface with marked points]
  \label{def:marked-surface}
  Let $\mathbf{S}$ be a connected oriented $2$-dimensional Riemann surface with
  boundary. We fix a finite set $\mathbf{M}$ of \emph{marked points} in the
  closure of $\mathbf{S}$. Marked points lying in the interior of $\mathbf{S}$
  are called \emph{punctures}. The pair $(\mathbf{S},\mathbf{M})$ is called a
  \emph{bordered surface with marked points} if the following additional
  technical conditions are satisfied.
  \begin{itemize}
   \item The set $\mathbf{M}$ is non-empty.
   \item The pair $(\mathbf{S}, \mathbf{M})$ is not
   \begin{itemize}
    \item a sphere with one or two punctures;
    \item a monogon with zero or one puncture;
    \item a digon without punctures;
    \item a triangle without punctures.
   \end{itemize}
  \end{itemize}
  Here, the term $n$-gon denotes a disk with $n$ marked points on its boundary.
  Moreover, sphere with three punctures is also often excluded.
 \end{definition}

 \begin{definition}[Arc]
  \label{def:arc}
  An arc $\gamma$ in a bordered surface with marked points
  $(\mathbf{S},\mathbf{M})$ is a curve in $\mathbf{S}$ such that
  \begin{itemize}
   \item its endpoints are marked points;
   \item $\gamma$ does not intersect itself, except that its endpoints may
    coincide;
   \item except for its endpoints, $\gamma$ is disjoint from $\mathbf{M}$ and
    from the boundary of $\mathbf{S}$;
   \item $\gamma$ is not contractible into $\mathbf{M}$ or into the boundary of
    $\mathbf{S}$.
  \end{itemize}
 \end{definition}

 We are interested in triangulations of $(\mathbf{S},\mathbf{M})$. Vaguely
 speaking, triangulation is a division of $\mathbf{S}$ into `triangles' by a
 series of `cuts'. Here, `triangles' are either disks with three marked points
 on their boundaries or, so-called \emph{self-folded} triangles, once-punctured
 monogons with an arc connecting the unique marked point to the unique puncture.
 See \cref{fig:self-folded-triangle}.
 \begin{figure}[ht]
  \centering
  \begin{tikzpicture}
   \node[circle,fill,inner sep=1pt] (a) at (0,0) {};
   \node[circle,fill,inner sep=1pt] (b) at (0,1) {};

   \draw (a) to node[midway,xshift=1.5mm] {$i$} (b);
   \draw (0,0.7) circle[radius=0.7];
  \end{tikzpicture}

  \caption{A self-folded triangle.}
  \label{fig:self-folded-triangle}
 \end{figure}

 \begin{definition}[Isotopy]
  \label{def:isotopy}
  Let $\gamma_1,\gamma_2$ be two arcs in $(\mathbf{S},\mathbf{M})$. An
  \emph{isotopy} between $\gamma_1$ and $\gamma_2$ is a homotopy $H$ between
  $\gamma_1$ and $\gamma_2$ such that $H(x,t)$ is an embedding for each fixed
  $t \in [0,1]$. Isotopy is an equivalence relation on the set of all arcs in
  $(\mathbf{S},\mathbf{M})$.
 \end{definition}
 In the following text, each arc in $(\mathbf{S},\mathbf{M})$ is considered up
 to isotopy.

 \begin{definition}[Compatibility of arcs]
  \label{def:compatibility-of-arcs}
  Two arcs in $(\mathbf{S},\mathbf{M})$ are called \emph{compatible} if they (up
  to isotopy) do not intersect each other in the interior of $\mathbf{S}$.
 \end{definition}

 \begin{proposition}
  \label{prop:triangulation-up-to-isotopy}
  Any collection of pairwise compatible arcs can be realized by curves in their
  respective isotopy classes which do not intersect in the interior of
  $\mathbf{S}$.
 \end{proposition}

 \begin{definition}[Ideal triangulation]
  \label{def:ideal-triangulation}
  A maximal collection of pairwise compatible arcs is called an \emph{ideal
  triangulation}. In fact, \cref{def:marked-surface} excludes all cases where
  $(\mathbf{S},\mathbf{M})$ cannot be triangulated. The arcs of an ideal
  triangulation cut $\mathbf{S}$ into \emph{ideal triangles}. The three sides of
  an ideal triangle need not be distinct, leading to self-folded triangles, and
  two triangles can share more than one side.
 \end{definition}

 The number of arcs in an ideal triangulation is an invariant of
 $(\mathbf{S},\mathbf{M})$ and is called the \emph{rank} of
 $(\mathbf{S},\mathbf{M})$ to emphasize the connection between these surfaces
 and cluster algebras of the same rank, to be introduced in the next section.

 The last concept we need to introduce is that of a \emph{flip} of an ideal
 triangulation. These basically entail swapping one diagonal for another in some
 quadrilateral of an ideal triangulation.

 \begin{definition}[Flip]
  A \emph{flip} in an ideal triangulation $T$ is a transformation which
  exchanges one arc $\gamma \in T$ for a different arc $\gamma'$ which, together
  with the rest of arcs in $T$, forms a new ideal triangulation $T'$.
 \end{definition}

 There is at most one way to flip and arc $\gamma$ in an ideal triangulation. If
 $\gamma$ is the `folded' side of a self-folded triangle (the segment $i$ in
 \cref{fig:self-folded-triangle}), then $\gamma$ cannot be flipped. In all other
 cases, removing $\gamma$ create a tetragonal face on $\mathbf{S}$, and the
 flipped arc $\gamma'$ is define to be its other diagonal.

 Of special import to the theory of cluster algebras is the following result.

 \begin{proposition}
  Any two ideal triangulations are related by a series of flips.
 \end{proposition}

 \begin{figure}[ht]
  \centering
  \begin{subfigure}{.3\textwidth}
   \centering
   \begin{tikzpicture}
    \node[fill,circle,inner sep=1pt] (a) at (18:1) {};
    \node[fill,circle,inner sep=1pt] (b) at (90:1) {};
    \node[fill,circle,inner sep=1pt] (c) at (162:1) {};
    \node[fill,circle,inner sep=1pt] (d) at (234:1) {};
    \node[fill,circle,inner sep=1pt] (e) at (306:1) {};

    \draw (a) -- (b);
    \draw (b) -- (c);
    \draw (c) -- (d);
    \draw (d) -- (e);
    \draw (e) -- (a);

    \draw (b) -- (e);
    \draw[Red] (c) -- (e);
   \end{tikzpicture}
  \end{subfigure}
  \begin{subfigure}{.3\textwidth}
   \centering
   \begin{tikzpicture}
    \node[fill,circle,inner sep=1pt] (a) at (18:1) {};
    \node[fill,circle,inner sep=1pt] (b) at (90:1) {};
    \node[fill,circle,inner sep=1pt] (c) at (162:1) {};
    \node[fill,circle,inner sep=1pt] (d) at (234:1) {};
    \node[fill,circle,inner sep=1pt] (e) at (306:1) {};

    \draw (a) -- (b);
    \draw (b) -- (c);
    \draw (c) -- (d);
    \draw (d) -- (e);
    \draw (e) -- (a);

    \draw (b) -- (e);
    \draw[Red] (d) -- (b);
   \end{tikzpicture}
  \end{subfigure}
  \caption{Two ideal triangulations of a pentagon, related by a flip.}
  \label{fig:flip-pentagon}
 \end{figure}

 \section{Cluster Algebras}
 \label{sec:cluster-algebras}

 Before we reach the definition of a cluster algebras, we must discuss quivers
 and their mutations. This section is an altered version of \cite{ot}, Section
 4. The original text defines cluster algebras using signed adjacency matrices
 instead. This approach lessens the notational burden but also introduces a
 concept we do not use elsewhere. Naturally, quivers and adjacency matrices are
 tightly related, with the former being completely determined by the latter
 after a choice of orientation of a single arrow.

 To each ideal triangulation $T$ of a bordered surface with marked points
 $(\mathbf{S},\mathbf{M})$ we associate a quiver $Q \coloneqq Q(T)$ in the
 following manner. For a fixed ideal triangulation, we label its $n$ arcs by
 natural numbers from $1$ to $n$, keeping in mind that this labelling is
 arbitrary. The set of vertices of $Q$ is then also $Q_0 \coloneqq
 \{1,\ldots,n\}$. Next, for each triangle $\Delta$ in $T$ which is not
 self-folded, we draw an arrow $i \to j$ if
 \begin{itemize}
  \item $i$ and $j$ are sides of $\Delta$ with $j$ following $i$ in clockwise
   order;
  \item $j$ is an arc `folded' by $l$, $i$ and $l$ are sides of $\Delta$ and $l$
   follows $i$ in clockwise order;
  \item $i$ is an arc `folded' by $l$, $l$ and $j$ are sides of $\Delta$ with
   $j$ following $l$ in clockwise order.
 \end{itemize}
 Finally, we remove all $2$-cycles (meaning a configuration of arrows like this
 \begin{tikzpicture}[baseline=-0.5ex]
  \node[circle,fill,inner sep=1pt] (a) at (0,0) {};
  \node[circle,fill,inner sep=1pt] (b) at (1,0) {};
  \draw[->,bend left=40] (a) to (b);
  \draw[<-,bend right=40] (a) to (b);
 \end{tikzpicture}).

 \begin{figure}[ht]
  \centering
  \begin{subfigure}{.4\textwidth}
   \centering
   \begin{tikzpicture}
    \clip (-2.1,-2.1) rectangle (2.1,2.1);
    \node[circle,fill,inner sep=1pt] (a) at (0,-2) {};
    \node[circle,fill,inner sep=1pt] (b) at (0,2) {};
    \draw (0,0) circle (2);

    \filldraw[Black,fill=LightGray] (1,0) circle (0.5);
    \node[circle,fill,inner sep=1pt] (c) at (0.552, 0.222) {};

    \node[circle,fill,inner sep=1pt] (d) at (-1,0) {};
    \draw (b) to node[midway,xshift=-2mm] {$1$} (d);
    \draw (b) to node[midway,xshift=2mm] {$3$} (c);
    \draw (a) to node[pos=0.6,xshift=2mm] {$2$} (b);

    \draw (b) .. controls ++(0,-3.5) and ++(-3.5,-3.5) .. (b);
    \node at (-1,-0.75) {$6$};
    
    \draw (a) .. controls ++(0.1,2) .. (c);
    \node at (0.2,-1) {$5$};
    
    \draw (b) .. controls ++(4,-2) and ++(-1,-3) .. (c);
    \node at (1,-1.25) {$4$};
   \end{tikzpicture}
  \end{subfigure}
  \begin{subfigure}{.4\textwidth}
   \centering
   \begin{tikzpicture}
    \clip (-0.1,-2.1) rectangle (4.1,2.1);
    \node (a) at (0,1) {$1$};
    \node (f) at (0,-1) {$6$};
    \node (b) at (1,0) {$2$};
    \node (c) at (2,1) {$3$};
    \node (e) at (2,-1) {$5$};
    \node (d) at (3,0) {$4$};
    
    \draw[->] (a) -- (b);
    \draw[->] (f) -- (b);
    \draw[->] (b) -- (c);
    \draw[->] (c) -- (d);
    \draw[->] (c) -- (e);
    \draw[->] (e) -- (b);
    \draw[->] (e) -- (d);
   \end{tikzpicture}
  \end{subfigure}
  \caption{An ideal triangulation of a once-punctured annulus and its quiver.}
  \label{fig:annulus-quiver}
 \end{figure}

 \begin{definition}[Quiver mutation]
   For a quiver $Q$, we construct the quiver $\mu_k(Q)$, called its
   \emph{mutation} at $k$-th vertex by
   \begin{itemize}
    \item reversing all arrows with $k$ as source or as target;
    \item adding an arrow $j \to i$ for each path $i \to k \to j$;
    \item deleting all $2$-cycles.
   \end{itemize}
 \end{definition}
 The mutation at $k$-th vertex is an involution of $Q$, that is, $\mu_k^2(Q) =
 Q$ for every vertex $k \in Q_0$. Now, we proceed to define initial seeds and
 cluster algebras.

 We fix a free abelian group $(\mathbb{P}, \cdot )$ on variables
 $y_1,\ldots,y_n$ and define an operation $ \oplus $ on $\mathbb{P}$ by the
 formula
 \[
  \prod_i y_i^{a_i} \oplus \prod_i y_i^{b_i} = \prod_i y_i^{\min(a_i,b_i)}.
 \]
 Let $\mathbb{Z}\mathbb{P}$ denote the group ring of $\mathbb{P}$ and
 $\mathbb{Q}\mathbb{P}$ the field of fractions of $\mathbb{Z}\mathbb{P}$.
 Finally, we let $\mathcal{F} \coloneqq \mathbb{Q}\mathbb{P}(x_1,\ldots,x_n)$ be
 the field of rational functions in variables $x_1,\ldots,x_n$ with coefficients
 in $\mathbb{Q}\mathbb{P}$.

 \begin{definition}[Initial seed]
  An \emph{initial seed} is a triple $(\mathbf{x},\mathbf{y},Q)$ consisting of
  the following data:
  \begin{enumerate}
   \item an $n$-tuple $\mathbf{x} = (x_1,\ldots,x_n)$ of variables of
    $\mathcal{F}$, the so-called \emph{initial cluster};
   \item an $n$-tuple $\mathbf{y} = (y_1,\ldots,y_n)$ of generators of
    $\mathbb{P}$, the so-called \emph{initial coefficients tuple};
   \item a quiver $Q$ without loops and $2$-cycles.
  \end{enumerate}
 \end{definition}

 \begin{definition}[Seed mutation]
  A \emph{seed mutation} $\mu(k)$ \emph{in direction} $k$ is a transformation of
  an initial seed $(\mathbf{x},\mathbf{y},Q)$ into a new seed
  $(\mathbf{x'},\mathbf{y'},Q')$ defined as follows:
  \begin{itemize}
   \item $\mathbf{x'}$ is the $n$-tuple of variables constructed by replacing
    the cluster variable $x_k$ in $\mathbf{x}$ by a new cluster variable $x_k'$
    defined by the \emph{exchange relation}
    \[
     x_kx_k' = \frac{1}{y_k \oplus 1}\left( y_k \prod_{i \to k} x_i +
     \prod_{i \leftarrow k} x_i \right);
    \]
   \item $\mathbf{y'} = (y_1',\ldots,y_n')$ is a new $n$-tuple of coefficients,
    where
    \[
     y_i' \coloneqq \begin{cases}
      y_k^{-1} & \text{ if } i = k;\\
      y_i \prod_{k \to i} y_k(y_k \oplus 1)^{-1}\prod_{k \leftarrow i} (y_k
      \oplus 1) & \text{ if } i \neq k.
     \end{cases}
    \]
   \item $Q'$ is the mutation of $Q$ at the $k$-th vertex.
  \end{itemize}
 \end{definition}
 Seed mutations are involutions, that is, $\mu_k^2(\mathbf{x},\mathbf{y},Q) =
 (\mathbf{x},\mathbf{y},Q)$.

 \begin{definition}[Cluster algebra]
  Let $(\mathbf{x},\mathbf{y},Q)$ be an initial seed and $\mathcal{X}$ be the
  set of all cluster variables generated by repeated mutation of
  $(\mathbf{x},\mathbf{y},Q)$. The \emph{cluster algebra} $\mathcal{A} \coloneqq
  \mathcal{A}(\mathbf{x},\mathbf{y},Q)$ is the $\mathbb{Z}\mathbb{P}$-subalgebra
  of $\mathcal{F}$ generated by $\mathcal{X}$.

  We say that a cluster algebra $\mathcal{A}(\mathbf{x},\mathbf{y},Q)$ is
  \begin{itemize}
   \item of \emph{finite type}, if the set $\mathcal{X}$ of cluster variables if
    finite;
   \item of \emph{finite mutation type}, if the number of quivers mutation
    equivalent (those $Q$ can mutate into) to $Q$ is finite;
   \item of \emph{acyclic type}, if $Q$ is mutation equivalent to a quiver
    without oriented cycles;
   \item of \emph{surface type}, if $Q$ is a quiver coming from a triangulation
    of a bordered surface with marked points.
  \end{itemize}
 \end{definition}

 We are particularly interested in quivers whose underlying graph is one of
 (simp\-ly-laced) Dynkin diagrams of type $A_n (n \geq 1)$, $D_n (n \geq 4)$ or
 $E_n (6 \leq n \leq 8)$ (see \cref{fig:dynkin-diagrams}). By~\cite{fst}, Theorem
 6.5, a cluster algebra $\mathcal{A}(\mathbf{x},\mathbf{y},Q)$ is of finite type
 if and only if the underlying graph of $Q$ is a disjoint union of the
 aforementioned Dynkin diagrams. In this particular case, it is also true (by
 \cite{fst}, Lemma 6.4) that quivers given by two different orientations of a
 Dynkin diagram are mutation equivalent, hence the structure of
 $\mathcal{A}(\mathbf{x},\mathbf{y},Q)$ is independent of the choice of
 orientation.

 \begin{figure}[ht]
  \centering
  \begin{subfigure}[b]{.6\textwidth}
   \centering
   \begin{tikzpicture}
    \node at (0,0) {$A_n$};
    \node[circle,fill,inner sep=1pt] (a) at (1,0) {};
    \node[circle,fill,inner sep=1pt] (b) at (2,0) {};
    \node[circle,fill,inner sep=1pt] (c) at (3,0) {};
    
    \draw (a) -- (b);
    \draw (b) -- (c);
    \draw (c) -- ++(0.5,0);

    \node at (4,0) {$\ldots$};
    \draw (4.5,0) -- ++(0.5,0);

    \node[circle,fill,inner sep=1pt] (d) at (5,0) {};
    \node[circle,fill,inner sep=1pt] (e) at (6,0) {};
    \draw (d) -- (e);
   \end{tikzpicture}
   \vspace{1em}
  \end{subfigure}
  \begin{subfigure}[b]{.6\textwidth}
   \centering
   \begin{tikzpicture}
    \node at (0,0) {$D_n$};
    \node[circle,fill,inner sep=1pt] (a1) at (1,0.5) {};
    \node[circle,fill,inner sep=1pt] (a2) at (1,-0.5) {};
    \node[circle,fill,inner sep=1pt] (b) at (2,0) {};
    \node[circle,fill,inner sep=1pt] (c) at (3,0) {};
    
    \draw (a1) -- (b);
    \draw (a2) -- (b);
    \draw (b) -- (c);
    \draw (c) -- ++(0.5,0);

    \node at (4,0) {$\ldots$};
    \draw (4.5,0) -- ++(0.5,0);

    \node[circle,fill,inner sep=1pt] (d) at (5,0) {};
    \node[circle,fill,inner sep=1pt] (e) at (6,0) {};
    \draw (d) -- (e);
    
   \end{tikzpicture}
   \vspace{1em}
  \end{subfigure}
  \begin{subfigure}[b]{.6\textwidth}
   \centering
   \begin{tikzpicture}
    \node at (0,0) {$E_n$};
    \node[circle,fill,inner sep=1pt] (a) at (1,0) {};
    \node[circle,fill,inner sep=1pt] (b) at (2,0) {};
    \node[circle,fill,inner sep=1pt] (c) at (3,0) {};
    \node[circle,fill,inner sep=1pt] (f) at (3,-0.5) {};
    
    \draw (a) -- (b);
    \draw (b) -- (c);
    \draw (c) -- ++(0.5,0);
    \draw (c) -- (f);

    \node at (4,0) {$\ldots$};
    \draw (4.5,0) -- ++(0.5,0);

    \node[circle,fill,inner sep=1pt] (d) at (5,0) {};
    \node[circle,fill,inner sep=1pt] (e) at (6,0) {};
    \draw (d) -- (e);
   \end{tikzpicture}
  \end{subfigure}
  \caption{Simply-laced Dynkin diagrams of types $A_n,D_n$ and $E_n$.}
  \label{fig:dynkin-diagrams}
 \end{figure}

 In seeking higher-dimensional geometric counterparts to cluster algebras of
 finite type, it is of course beneficial to -- at least at first -- focus on
 those that are also of surface type. Here, one has a starting idea as to what
 the higher-dimensional object in question should be. The only such cluster
 algebras are of type $A_n$ and $D_n$.  Quivers given by orientations of $E_n$
 do not arise from any bordered surface with marked points.  

 Moreover, based on the results in~\cite{bma}, for purposes of categorisation,
 one need not consider the entire class of cluster algebras with a chosen Dynkin
 quiver. The combinatorial properties of $\mathcal{A}(\mathbf{x},\mathbf{y},Q)$
 are in fact governed entirely by so-called \emph{decorated representations} of
 the path algebra of $Q$ (see~\cite{bma}, Section 2).

 Hence, studying the higher Auslander algebras (to be introduced promptly) of
 path algebras of Dynkin quivers appears to be a sensible endeavour in this
 direction.

 By a considerable extension, one might also consider studying
 higher-dimensional variants of cluster algebras of `affine' type, whose quiver
 is one of so-called \emph{affine} Dynkin diagrams. For these, however, Lemma
 6.4 from~\cite{fst} does not apply and thus the choice of orientation matters.
 Furthermore, the path algebras of such diagrams are in general
 representation-infinite leading to caveats in applying the higher Auslander
 theory developed in~\cite{iyama}.

 \begin{figure}[ht]
  \centering
  \begin{subfigure}[b]{.6\textwidth}
   \centering
   \begin{tikzpicture}
    \node at (0,0) {$\tilde{A}_n$};
    \node[circle,fill,inner sep=1pt] (a) at (1,0) {};
    \node[circle,fill,inner sep=1pt] (b) at (2,0) {};
    \node[circle,fill,inner sep=1pt] (c) at (3,0) {};
    
    \draw (a) -- (b);
    \draw (b) -- (c);
    \draw (c) -- ++(0.5,0);

    \node at (4,0) {$\ldots$};
    \draw (4.5,0) -- ++(0.5,0);

    \node[circle,fill,inner sep=1pt] (d) at (5,0) {};
    \node[circle,fill,inner sep=1pt] (e) at (6,0) {};
    \draw (d) -- (e);

    \node[circle,fill,inner sep=1pt] (f) at (3.5,-1) {};
    \draw (a) -- (f);
    \draw (e) -- (f);
   \end{tikzpicture}
   \vspace{1em}
  \end{subfigure}
  \begin{subfigure}[b]{.6\textwidth}
   \centering
   \begin{tikzpicture}
    \node at (0,0) {$\tilde{D}_n$};
    \node[circle,fill,inner sep=1pt] (a1) at (1,0.5) {};
    \node[circle,fill,inner sep=1pt] (a2) at (1,-0.5) {};
    \node[circle,fill,inner sep=1pt] (b) at (2,0) {};
    \node[circle,fill,inner sep=1pt] (c) at (3,0) {};
    
    \draw (a1) -- (b);
    \draw (a2) -- (b);
    \draw (b) -- (c);
    \draw (c) -- ++(0.5,0);

    \node at (4,0) {$\ldots$};
    \draw (4.5,0) -- ++(0.5,0);

    \node[circle,fill,inner sep=1pt] (d) at (5,0) {};
    \node[circle,fill,inner sep=1pt] (e1) at (6,0.5) {};
    \node[circle,fill,inner sep=1pt] (e2) at (6,-0.5) {};
    \draw (d) -- (e1);
    \draw (d) -- (e2);
   \end{tikzpicture}
   \vspace{1em}
  \end{subfigure}
  \caption{Affine Dynkin diagrams of types $\tilde{A}_n$ and $\tilde{D}_n$.}
  \label{fig:affine-dynkin-diagrams}
 \end{figure}

 \section{Higher Auslander Theory}
 \label{sec:higher-auslander-theory}

 In this section, we summarize results from~\cite{iyama}, Section 1. We fix a
 finite-dimensional algebra $\Lambda$ over a field $k$.

 \begin{definition}[$d$-cluster tilting module]
  A module $M \in \mod \Lambda$ is called \emph{$d$-cluster tilting} if
  \[
   \add M = \{X \in \mod \Lambda \mid \ext^{i}_{\Lambda}(X,M) = 0 \; \forall i
   \in \{1,\ldots,d-1\}\}.
  \]
 \end{definition}
 We note that a $1$-cluster tilting module is just an additive generator of the
 category $\mod \Lambda$.
 \begin{definition}[$d$-Auslander algebra]
  \leavevmode
  \begin{enumerate}
   \item An algebra $\Lambda$ is called \emph{$d$-representation-finite} if
    $\gldim\Lambda<\infty$ and $\Lambda$ has a $d$-cluster tilting module.
   \item Let $\Lambda$ be a $d$-representation-finite algebra and $M$ its
    $d$-cluster tilting module. We call $\mathrm{End}_{\Lambda}(M)$ the
    \emph{$d$-Auslander algebra} of $\Lambda$.
  \end{enumerate}
 \end{definition}

 By~\cite{iyama}, Theorem 1.6, if $\Lambda$ is $d$-representation-finite then
 its $d$-cluster tilting module is unique up to multiplicity. A
 $1$-representation-finite algebra is simply called representation-finite.

 One of the main results in~\cite{iyama} concerns an iterative construction of
 $d$-Auslander algebras of a representation-finite hereditary algebra $\Lambda$.
 In this particular case, it turns out that $\mathrm{End}_{\Lambda}(M)$, where
 $M$ is a cluster tilting module in $\mod\Lambda$, is a
 $2$-representation-finite algebra with its own $2$-cluster tilting module. This
 allows us to construct a $d$-Auslander algebra of $\Lambda$ for any $d \geq 1$.

 In~\cite{iyama}, Definition 6.11 and Theorem 6.12 describes the $d$-Auslander
 algebras of representation-finite hereditary algebras by giving their
 Auslander-Reiten quivers with relations. For example, let $\Lambda \coloneqq
 kD_4$ where $k$ is a field. We denote the $d$-Auslander algebra of $D_4$ by
 $D_4^{(d)}$ and $D_4^{(0)} \coloneqq \Lambda$. The quivers of $D_4^{(1)}$ and
 $D_4^{(2)}$ are depicted in~\cref{fig:quivers-of-D}. We use the same labels for
 vertices as Iyama in~\cite{iyama}. Iyama also introduces different `kinds' of
 arrows in these quivers. We only distinguish them by style and colour to avoid
 having to introduce auxiliary notation.
 \begin{figure}[ht]
  \centering
  \begin{subfigure}[b]{\textwidth}
   \centering
   \begin{tikzpicture}
    \node (10) at (1,0) {$10$};
    \node (20) at (0,-1) {$20$};
    \node (30) at (0,0) {$30$};
    \node (40) at (0,1) {$40$};
    
    \node (11) at (3.5,0) {$11$};
    \node (21) at (2.5,-1) {$21$};
    \node (31) at (2.5,0) {$31$};
    \node (41) at (2.5,1) {$41$};

    \node (12) at (6,0) {$12$};
    \node (22) at (5,-1) {$22$};
    \node (32) at (5,0) {$32$};
    \node (42) at (5,1) {$42$};

    \foreach \d in {0,1,2} {
     \foreach \x in {2,3,4} {
      \draw[->] (\x\d) -- (1\d);
     }
    }

    \foreach \d in {1,2} {
     \foreach \x in {2,3,4} {
      \tikzmath{\e = \d - 1;}
      \draw[->,dashed,thick,ForestGreen] (1\e) -- (\x\d);
     }
    }
   \end{tikzpicture}
   \caption{The AR quiver of $D_4^{(1)}$.}
  \end{subfigure}
  \begin{subfigure}[b]{\textwidth}
   \vspace*{1em}
   \centering
   \begin{tikzpicture}
    \pgfdeclarelayer{background}
    \pgfsetlayers{background,main}
    \begin{pgfonlayer}{main}
     \node (100) at (1,0) {$100$};
     \node (200) at (0,-1) {$200$};
     \node (300) at (0,0) {$300$};
     \node (400) at (0,1) {$400$};

     \node (110) at (3.5,0) {$110$};
     \node (210) at (2.5,-1) {$210$};
     \node (310) at (2.5,0) {$310$};
     \node (410) at (2.5,1) {$410$};

     \node (120) at (6,0) {$120$};
     \node (220) at (5,-1) {$220$};
     \node (320) at (5,0) {$320$};
     \node (420) at (5,1) {$420$};

     \node (101) at (2.25,-3) {$101$};
     \node (201) at (1.25,-4) {$201$};
     \node (301) at (1.25,-3) {$301$};
     \node (401) at (1.25,-2) {$401$};

     \node (111) at (4.75,-3) {$111$};
     \node (211) at (3.75,-4) {$211$};
     \node (311) at (3.75,-3) {$311$};
     \node (411) at (3.75,-2) {$411$};

     \node (102) at (3.5,-6) {$102$};
     \node (202) at (2.5,-7) {$202$};
     \node (302) at (2.5,-6) {$302$};
     \node (402) at (2.5,-5) {$402$};
    \end{pgfonlayer}
    \begin{pgfonlayer}{background}
     \foreach \d in {0,1,2} {
      \foreach \x in {2,3,4} {
       \draw[->] (\x\d0) -- (1\d0);
      }
     }
     \foreach \d in {0,1} {
      \foreach \x in {2,3,4} {
       \draw[->] (\x\d1) -- (1\d1);
      }
     }
     \foreach \x in {2,3,4} {
      \draw[->] (\x02) -- (102);
     }

     \foreach \d in {1,2} {
      \foreach \x in {2,3,4} {
       \tikzmath{\e = \d - 1;}
       \draw[->,dashed,thick,ForestGreen] (1\e0) -- (\x\d0);
      }
     }
     \foreach \x in {2,3,4} {
      \draw[->,dashed,thick,ForestGreen] (101) -- (\x11);
     }

     \foreach \x in {1,2,3,4} {
      \draw[thick,->,dotted,Blue] (\x10) -- (\x01);
      \draw[thick,->,dotted,Blue] (\x20) -- (\x11);
      \draw[thick,->,dotted,Blue] (\x11) -- (\x02);
     }
    \end{pgfonlayer}
   \end{tikzpicture}
   \caption{The AR quiver of $D_4^{(2)}$.}
  \end{subfigure}
  \caption{The Auslander-Reiten quivers of $D_4^{(1)}$ and $D_4^{(2)}$.}
  \label{fig:quivers-of-D}
 \end{figure}

 As was mentioned, the graphs of Dynkin type $A_n$ and $D_n$ describe bordered
 surfaces with marked points. In the case of $A_n$, it is an unpunctured polygon
 with $n + 3$ vertices and in the case of $D_n$ it is a once-punctured polygon
 with $n$ vertices. Both assertions are proven in~\cite{fst} but are also easily
 verifiable for instance from~\cref{fig:triangulations-An-Dn}.

\begin{figure}[ht]
 \centering
 \begin{subfigure}[b]{\textwidth}
  \centering
  \begin{subfigure}[c]{.3\textwidth}
   \centering
   \begin{tikzpicture}
    \foreach \angle/\name in {0/A,60/B,120/C,180/D,240/E,300/F} {
     \node[circle,fill,inner sep=1pt] (\name) at (\angle:1) {};
    }
    \draw (0,0) circle (1);
    \draw (C) to node[midway,xshift=-1mm] {$1$} (E);
    \draw (E) to node[midway,xshift=-1mm,yshift=1mm] {$2$} (B);
    \draw (B) to node[midway,xshift=1mm] {$3$} (F);
   \end{tikzpicture}
  \end{subfigure}
  \begin{subfigure}[c]{.3\textwidth}
   \centering
   \begin{tikzpicture}
    \node (1) at (0,0) {$1$};
    \node (2) at (1,0) {$2$};
    \node (3) at (2,0) {$3$};

    \draw[->] (2) -- (1);
    \draw[->] (2) -- (3);
   \end{tikzpicture}
  \end{subfigure}
  \caption{Ideal triangulation of a hexagon and the corresponding orientation
   of $A_3$.}
 \end{subfigure}
 \begin{subfigure}[b]{\textwidth}
  \vspace{1em}
  \centering
  \begin{subfigure}[c]{.3\textwidth}
   \centering
   \begin{tikzpicture}
    \foreach \an/\n in {0/A,72/B,160/C,216/D,250/E} {
     \node[circle,fill,inner sep=1pt] (\n) at (\an:1) {};
    }
    \node[circle,fill,inner sep=1pt] (P) at (36:0.6) {};
    \draw (0,0) circle (1);
    \draw (C) to node[pos=0.2,xshift=1mm,yshift=1mm] {$1$} (E);
    \draw (A) to node[midway,xshift=-1mm,yshift=-1.2mm] {$5$} (P);
    \draw (B) to node[midway,xshift=-1.5mm] {$4$} (P);
    \draw (B) .. controls ++(-0.75,-0.5) and ++(-0.5,-0.5) .. (A);
    \node at (-0.1,0.5) {$3$};
    \draw (B) .. controls ++(-1,0.1) and ++(0,-0.1) .. (E);
    \node at (-0.27,-0.5) {$2$};
   \end{tikzpicture}
  \end{subfigure}
  \begin{subfigure}[c]{.3\textwidth}
   \centering
   \begin{tikzpicture}
    \node (1) at (0,0) {$1$};
    \node (2) at (1,0) {$2$};
    \node (3) at (2,0) {$3$};
    \node (4) at (3,0.5) {$4$};
    \node (5) at (3,-0.5) {$5$};
    
    \draw[->] (2) -- (1);
    \draw[->] (2) -- (3);
    \draw[->] (3) -- (4);
    \draw[->] (5) -- (3);
   \end{tikzpicture}
  \end{subfigure}
  \caption{Ideal triangulation of a once-punctured pentagon and the
  corresponding orientation of $D_5$.}
 \end{subfigure}
 \caption{Triangulations of an unpunctured polygon on $n+3$ vertices and a
  once-punctured polygon on $n$ vertices defining orientations of $A_n$ and
  $D_n$.}
 \label{fig:triangulations-An-Dn}
 \end{figure}

 In the face of this correspondence, one might hope to find similar connection
 between unpunctured, 
 \bibliography{refs.bib}
\end{document}
