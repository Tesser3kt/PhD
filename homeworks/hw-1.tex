\documentclass[a4paper,11pt]{article}

% Colors %
\usepackage[dvipsnames]{xcolor}

% Page Layout %
\usepackage[margin=1.5in]{geometry}

% Fancy Headers %
\usepackage{fancyhdr}
\fancyhf{}
\cfoot{\thepage}
\rhead{Adam Klepáč}
\renewcommand{\headrulewidth}{0pt}
\setlength{\headheight}{16pt}

% Math
\usepackage{mathtools}
\usepackage{amssymb}
\usepackage{faktor}
\usepackage{import}
\usepackage{caption}
\usepackage{subcaption}
\usepackage{wrapfig}
\usepackage{tikz}
\usepackage{tikz-cd}

% Theorems
\usepackage{amsthm}
\usepackage{thmtools}

% Title %
\title{\Huge\textsf{Assignment I}\\
 \Large\textsf{Dualities In Triangulated Categories}
 \author{}
 \date{}
}

% Table of Contents %
\usepackage{hyperref}
\hypersetup{
 colorlinks=true,
 linktoc=all,
 linkcolor=blue
}

% Tables %
\usepackage{booktabs}

% Enumerate %
\usepackage{enumitem}

% Operators %
\DeclareMathOperator{\Ker}{Ker}
\DeclareMathOperator{\Img}{Im}
\DeclareMathOperator{\End}{End}
\DeclareMathOperator{\Aut}{Aut}
\DeclareMathOperator{\Inn}{Inn}
\DeclareMathOperator{\Ob}{Ob}
\DeclareMathOperator{\op}{op}

% Common operators %
\newcommand{\R}{\mathbb{R}}
\newcommand{\N}{\mathbb{N}}
\newcommand{\Z}{\mathbb{Z}}
\newcommand{\Q}{\mathbb{Q}}
\newcommand{\C}{\mathbb{C}}

% American Paragraph Skip %
\setlength{\parindent}{0pt}
\setlength{\parskip}{1em}

% Document %
\pagestyle{fancy}
\begin{document}

\maketitle
\thispagestyle{fancy}

\begin{enumerate}[label=(\arabic*)]
 \item Consider the arrow $X \overset{f}{\longrightarrow} Y$. By the axiom
  (TR1), there exists a distinguished triangle
 \begin{center}
  \begin{tikzcd}
   X \arrow[r, "f"] & Y \arrow[r,"i"] & W \arrow [r,"j"] & \Sigma X.
  \end{tikzcd}
 \end{center}
 Also, by the axiom (TR3), there exists an arrow $\alpha:W \to Z \oplus Z'$,
 such that the diagram
 \begin{center}
  \begin{tikzcd}[row sep=large]
   X \arrow[r, "f"] \arrow[d,"{\begin{psmallmatrix} 1_X \\ 0
   \end{psmallmatrix}}"] & Y \arrow[r,"i"] \arrow[d,"{\begin{psmallmatrix} 1_Y
   \\ 0 \end{psmallmatrix}}"] & W \arrow [r,"j"] \arrow[d,dashed,"\exists
   \alpha"] & \Sigma X \arrow[d,"{\begin{psmallmatrix} 1_{\Sigma X} \\ 0
   \end{psmallmatrix}}"]\\ X \oplus X' \arrow[r, "f \oplus f'"] & Y \oplus Y'
   \arrow[r, "g \oplus g'"] & Z \oplus Z' \arrow[r, "h \oplus h'"] & \Sigma X
   \oplus \Sigma X' 
  \end{tikzcd}
 \end{center}
 commutes, where $1_A$ denotes the identity arrow of $A$ and
 $\begin{psmallmatrix} 1_A \\ 0 \end{psmallmatrix}$ the (unique) inclusion
 arrow $A \hookrightarrow A \oplus A'$ for $A \in \{X,Y,\Sigma X\}$.

 Finally, consider the diagram
 \begin{center}
  \begin{tikzcd}[row sep=large]
   X \arrow[r, "f"] \arrow[d,"{\begin{psmallmatrix} 1_X \\ 0
   \end{psmallmatrix}}"] & Y \arrow[r,"i"] \arrow[d,"{\begin{psmallmatrix} 1_Y
   \\ 0 \end{psmallmatrix}}"] & W \arrow [r,"j"] \arrow[d,dashed,"\exists
   \alpha"] & \Sigma X \arrow[d,"{\begin{psmallmatrix} 1_{\Sigma X} \\ 0
   \end{psmallmatrix}}"]\\ X \oplus X' \arrow[r, "f \oplus f'"] \arrow[d,"(1_X
   \; 0)"] & Y \oplus Y' \arrow[r, "g \oplus g'"] \arrow[d,"(1_Y \; 0)"] & Z
   \oplus Z' \arrow[r, "h \oplus h'"] \arrow[d, "(1_Z \; 0)"] & \Sigma X \oplus
   \Sigma X' \arrow[d,"(1_{\Sigma X} \; 0)"] \\ X \arrow[r, "f"] & Y
   \arrow[r,"g"] & Z \arrow [r,"h"] & \Sigma X,
  \end{tikzcd}
 \end{center}
 where $(1_A \; 0)$ denote the projection arrows. The top rectangle is
 commutative by (TR3) and the commutativity of the bottom rectangle is obvious.
 Thus, the diagram is a morphism of triangles. Moreover, the maps $(1_A \; 0)
 \circ \begin{psmallmatrix} 1_A \\ 0 \end{psmallmatrix} = 1_A$ are all
 isomorphisms so \textbf{Lemma 2.14.} asserts that also $(1_Z \; 0) \circ
 \alpha$ is an isomorphism which in turn means that $T$ is isomorphic to the
 distinguished triangle
 \begin{center}
  \begin{tikzcd}
   X \arrow[r, "f"] & Y \arrow[r,"i"] & W \arrow [r,"j"] & \Sigma X
  \end{tikzcd}
 \end{center}
 making it a distinguished triangle as well, by (TR0).
 
 The argument for the distinction of $T'$ is nearly identical.
\item 
 \begin{enumerate}[label=(\roman*)]
 \item By definition of $\mathsf{K}(\Z)$, there exists a distinguished triangle
  \begin{center}
   \begin{tikzcd}
    \Z / 2 \arrow[r, "\cdot 2"] & \Z / 4 \arrow[r, "i"] & C(\cdot 2) \arrow[r,
    "c_{\cdot 2}"] & \Sigma(\Z / 2),
   \end{tikzcd}
  \end{center}
  where $i: \Z / 4 \hookrightarrow C( \cdot 2)$ is the canonical inclusion for
  $C( \cdot 2)$ is the complex $(\Z / 2 \overset{-( \cdot
  2)}{\longrightarrow} \Z / 4)$ concentrated in degrees $1$ and $0$.

  Assume for contradiction that there also exists a map $h: \Z / 2 \to \Sigma(\Z
  / 2)$ such that
  \begin{center}
   \begin{tikzcd}
    \Z / 2 \arrow[r, "\cdot 2"] & \Z / 4 \arrow[r, "\bmod 2"] & \Z / 2 \arrow[r,
    "h"] & \Sigma(\Z / 2)
   \end{tikzcd}
  \end{center}
  is a distinguished triangle. By (TR3), there exists a map $\alpha: C( \cdot 2)
  \to \Z / 2$ making the diagram
  \begin{center}
   \begin{tikzcd}
    \Z / 2 \arrow[r, "\cdot 2"] \arrow[d,"1"] & \Z / 4 \arrow[r, "i"]
    \arrow[d,"1"] & C(\cdot 2) \arrow[r, "c_{\cdot 2}"] \arrow[d, dashed,
    "\exists \alpha"] & \Sigma(\Z / 2) \arrow[d,"1"] \\
    \Z / 2 \arrow[r, "\cdot 2"] & \Z / 4 \arrow[r, "\bmod 2"] & \Z / 2 \arrow[r,
    "h"] & \Sigma(\Z / 2)
   \end{tikzcd}
  \end{center}
  commute. By commutativity of the central square, we get that the map $\alpha
  \circ i - \bmod \; 2$ is null-homotopic, which in the case of complexes
  concentrated in degree $0$ simply means that $\alpha \circ i = \bmod \, 2$. As
  $i$ is just $\Z / 4 \overset{1}{\longrightarrow} \Z / 4$ in degree $0$, it
  follows that $\alpha = \bmod \; 2$. By definition of $C( \cdot 2)$, the map
  $c_{ \cdot 2}:C( \cdot 2) \to \Sigma(\Z / 2)$ is the map $\Z / 2
  \overset{1}{\longrightarrow} \Z / 2$ in degree $1$. The commutativity of the
  rightmost square implies that $c_{ \cdot 2} = h \circ \alpha = h \circ \bmod
  \; 2$. Nonetheless, this is a contradiction as $h \circ \bmod \; 2 = 0$
  because
  \begin{center}
   \begin{tikzcd}
    \Z / 2 \arrow[r, "\cdot 2"] & \Z / 4 \arrow[r, "\bmod 2"] & \Z / 2 \arrow[r,
    "h"] & \Sigma(\Z / 2)
   \end{tikzcd}
  \end{center}
  is a triangle. It follows that there indeed doesn't exist a distinguished
  triangle of the above form.
 \item 
 \end{enumerate}
\end{enumerate}

\end{document}
