\documentclass[a4paper,oneside,svgnames]{amsart}

\usepackage[T1]{fontenc}
\usepackage[utf8]{inputenc}

\usepackage[margin=1in]{geometry}
\usepackage{xcolor}
\usepackage[colorlinks=true,citecolor=Green,urlcolor=Black]{hyperref}
\usepackage[backend=biber,style=alphabetic,sorting=ynt]{biblatex}
\addbibresource{refs.bib}
\usepackage{tikz}
\usetikzlibrary{arrows,math}
\usepackage[noabbrev]{cleveref}
\usepackage{caption}
\usepackage{subcaption}
\usepackage{mathtools}
\renewcommand{\thesubsection}{\arabic{subsection}}

% Theorems
\theoremstyle{plain}
\newtheorem{theorem}{Theorem}[section]
\newtheorem{proposition}[theorem]{Proposition}
\newtheorem{lemma}[theorem]{Lemma}

\theoremstyle{definition}
\newtheorem{definition}[theorem]{Definition}

\let\hom\relax
\let\mod\relax
\DeclareMathOperator{\ext}{Ext}
\DeclareMathOperator{\add}{add}
\DeclareMathOperator{\hom}{Hom}
\DeclareMathOperator{\End}{End}
\DeclareMathOperator{\mod}{mod}
\DeclareMathOperator{\soc}{soc}
\DeclareMathOperator{\rad}{rad}
\DeclareMathOperator{\per}{per}
\DeclareMathOperator{\gldim}{gl.dim}

\newcommand{\C}{\mathbb{C}}
\newcommand{\Z}{\mathbb{Z}}

\title{Combinatorial Properties of Higher Cluster Categories}
\author{Adam Klepáč}

\begin{document}
 \maketitle
 \section*{Research Objectives}
 \setcounter{section}{1}

 The main objective of this project is to understand and describe the
 combinatorics of higher-dimension\-al analogues of the path algebras (in the
 sense of higher representation theory developed by Iyama in~\cite{iya}) of
 affine Dynkin quivers and their associated generalized cluster categories (in
 the sense described in~\cite[Section~5]{ot}, building upon~\cite{amiot1} and
 \cite{amiot2}, or~\cite{guo}), possibly and hopefully aiding in an eventual
 broader comprehension of the combinatorial structure of
 $n$-representation-infinite algebras (see~\cite{hio}).

 The path towards the goal of this project can be divided into several steps,
 ordered chronologically and quite possibly also by level of difficulty.
 
 \subsection{Combinatorial description of higher Auslander algebras of type $D$
 and $E$ and the associated cluster categories}

 The first step entails expanding upon the research done on the higher Auslander
 algebras of type $A$ in~\cite{ot} with the auxiliary goal of deepening my
 understanding of higher representation theory and the theory of cluster
 categories.

 Dynkin quivers of type $D$ arise (as well as those of type $A$) from
 triangulations of bordered surfaces with marked points -- in this case the
 surface in question is a disk with finite number of marked points on its
 boundary and one marked point in its interior. The most natural
 higher-dimensional structure whose triangulations should describe basic tilting
 modules of their path algebras and also the basic cluster tilting objects of
 their associated cluster categories is the cyclic polytope with a distinguished
 point in its interior. As the higher Auslander algebras of such algebras are
 $n$-representation-finite, the necessary theory is already in place and it
 remains to successfully describe (in combinatorial terms) the tilting objects
 in question.

 Quivers of Dynkin type $E$ do not arise from triangulations of bordered
 surfaces and thus there are no obvious hints as to which combinatorial
 structure captures the nature of the tilting modules of the higher Auslander
 algebras of their path algebras. Nonetheless, as there are only three Dynkin
 diagrams of type $E$, at worst such structures can be found on a case-by-case
 basis.

 \subsection{Generalization of the theory of higher Auslander algebras to path
 algebras of quivers of affine Dynkin type}
 \label{step-2}

 The iterative construction of higher Auslander algebras detailed in~\cite{iya}
 hinges on the property of the `starting' algebra being representation-finite
 and hereditary and thus having a (up to multiplicity unique) cluster tilting
 object. In order to study representation-infinite path algebras from the same
 viewpoint, a generalization of this theory which compensates for the
 non-existence of such an object is paramount. An ideal generalization would be
 one that is also symmetric with respect to changes in orientation of the
 original quiver which now influences the structure of the associated cluster
 algebras and cluster categories.

 The construction of $n$-represention-infinite algebras of type $\tilde{A}$
 from~\cite[Section~5]{hio} seems promising but still fully depends on the
 chosen orientation of the graph $\tilde{A}_n$ and in degree 0 yields path
 algebras of the resulting quiver only in cases when the chosen orientation is
 acyclic. I suspect, however, that the latter will be a prevailing issue and
 path algebras of the (exactly two) quivers of type $\tilde{A}$ with cyclic
 orientation will need to be either excluded or treated separately.

 \subsection{Combinatorial description of the generalized higher Auslander
 algebras of quivers of affine Dynkin type}
 \label{step-3}

 Continuing in the same vein, I intend to find or construct a combinatorial
 structure which would encapsulate information about the (primarily tilting)
 modules of the generalized higher Auslander algebras from step~\ref{step-2}.
 Such a structure will undoubtedly fail to be finite but I find it unreasonable
 to try to predict what such a structure could look like considering I have not
 yet conceptualized, much less studied, said generalization.

 The division of the category of finitely generated modules over
 $n$-representation-infinite algebras into $n$-preprojective, $n$-regular and
 $n$-preinjective parts from~\cite[Theorem 4.18]{hio} is bound to prove itself
 useful in this endeavour as it already allowed a deeper delve into the
 combinatorics of the AR quivers of these algebras (\cite{gll}).

 \subsection{Construction of a generalized cluster category associated to the
 generalized higher Auslander algebras of affine Dynkin type}

 The final step is to construct a generalized cluster category in a manner
 similar to the one employed in~\cite[Section~5]{ot} for
 $n$-representation-infinite algebras or, at least, for the generalized higher
 Auslander algebras of affine Dynkin type from step~\ref{step-2}. It is quite
 possible that the
 % TODO
 which of the two related approaches (\cite{amiot1} or \cite{guo}) is preferable.
 In any case, the resulting cluster category must be $n$-Calabi-Yau, admit an
 $m$-cluster tilting object ($n$ and $m$ need not be equal but dependant), its
 cluster tilting objects should in the 2-dimensional case correspond to clusters
 of the connected cluster algebra and the combinatorial description of its
 cluster tilting objects must be compatible (in the vague sense of `bearing a
 related structure') with the description from step~\ref{step-3}.

\end{document}
