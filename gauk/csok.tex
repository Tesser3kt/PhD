\documentclass[a4paper,oneside,svgnames]{amsart}

\usepackage[T1]{fontenc}
\usepackage[utf8]{inputenc}

\usepackage[margin=1in]{geometry}
\usepackage{xcolor}
\usepackage[colorlinks=true,citecolor=Green,urlcolor=Black]{hyperref}
\usepackage[backend=biber,style=alphabetic,sorting=ynt]{biblatex}
\addbibresource{refs.bib}
\usepackage{tikz}
\usetikzlibrary{arrows,math}
\usepackage[noabbrev]{cleveref}
\usepackage{caption}
\usepackage{subcaption}
\usepackage{mathtools}

% Theorems
\theoremstyle{plain}
\newtheorem{theorem}{Theorem}[section]
\newtheorem{proposition}[theorem]{Proposition}
\newtheorem{lemma}[theorem]{Lemma}

\theoremstyle{definition}
\newtheorem{definition}[theorem]{Definition}

\let\hom\relax
\let\mod\relax
\DeclareMathOperator{\ext}{Ext}
\DeclareMathOperator{\add}{add}
\DeclareMathOperator{\hom}{Hom}
\DeclareMathOperator{\End}{End}
\DeclareMathOperator{\mod}{mod}
\DeclareMathOperator{\soc}{soc}
\DeclareMathOperator{\rad}{rad}
\DeclareMathOperator{\per}{per}
\DeclareMathOperator{\gldim}{gl.dim}

\newcommand{\C}{\mathbb{C}}
\newcommand{\Z}{\mathbb{Z}}

\title{Combinatorial Properties of Higher Cluster Categories}
\author{Adam Klepáč}

\begin{document}
 \maketitle
 \section*{Current State of Knowledge}

 In~\cite{fz1}, Fomin and Zelenivsky introduce \emph{cluster algebras} to
 provide an algebraic-combinatorial description of the structure of dual
 canonical bases in coordinate rings of varieties associated to semisimple
 algebraic groups. Such descriptions were indeed found in several cases
 (see~\cite{fzb}, \cite{s}). Applications of cluster algebras have since been
 discovered in, for instance, Poisson geometry (\cite{gsv}), discrete dynamical
 systems (e.g.~\cite{fz3}), higher Teichmüller spaces (\cite{fg}) or commutative
 and non-commutative algebraic geometry (\cite{ifz}, \cite{ks} or~\cite{gmn}).

 Of particular interest to me are cluster algebras of \emph{finite type}. Those
 were fully classified in~\cite{fz4} -- a cluster algebra $\mathcal{A}(Q)$ is of
 finite type if and only if the underlying graph $\Delta$ of the associated
 quiver $Q$ is a (simply-laced) Dynkin diagram
 (figure~\ref{fig:dynkin-diagrams}). It was shown in~\cite{mrz} that in this
 case, the combinatorics of $\mathcal{A}(Q)$ can be obtained from the category
 of \emph{decorated representations} of $Q$. However, with the construction of a
 \emph{cluster category} in~\cite{bmrrt}, emerged a more symmetric way of
 describing the combinatorics of $\mathcal{A}(Q)$.
 \begin{figure}[ht]
  \centering
  \begin{subfigure}[b]{.6\textwidth}
   \centering
   \begin{tikzpicture}
    \node at (0,0) {$A_n$};
    \node[circle,fill,inner sep=1pt] (a) at (1,0) {};
    \node[circle,fill,inner sep=1pt] (b) at (2,0) {};
    \node[circle,fill,inner sep=1pt] (c) at (3,0) {};
    
    \draw (a) -- (b);
    \draw (b) -- (c);
    \draw (c) -- ++(0.5,0);

    \node at (4,0) {$\ldots$};
    \draw (4.5,0) -- ++(0.5,0);

    \node[circle,fill,inner sep=1pt] (d) at (5,0) {};
    \node[circle,fill,inner sep=1pt] (e) at (6,0) {};
    \draw (d) -- (e);
   \end{tikzpicture}
   \vspace{1em}
  \end{subfigure}
  \begin{subfigure}[b]{.6\textwidth}
   \centering
   \begin{tikzpicture}
    \node at (0,0) {$D_n$};
    \node[circle,fill,inner sep=1pt] (a1) at (1,0.5) {};
    \node[circle,fill,inner sep=1pt] (a2) at (1,-0.5) {};
    \node[circle,fill,inner sep=1pt] (b) at (2,0) {};
    \node[circle,fill,inner sep=1pt] (c) at (3,0) {};
    
    \draw (a1) -- (b);
    \draw (a2) -- (b);
    \draw (b) -- (c);
    \draw (c) -- ++(0.5,0);

    \node at (4,0) {$\ldots$};
    \draw (4.5,0) -- ++(0.5,0);

    \node[circle,fill,inner sep=1pt] (d) at (5,0) {};
    \node[circle,fill,inner sep=1pt] (e) at (6,0) {};
    \draw (d) -- (e);
    
   \end{tikzpicture}
   \vspace{1em}
  \end{subfigure}
  \begin{subfigure}[b]{.6\textwidth}
   \centering
   \begin{tikzpicture}
    \node at (0,0) {$E_n$};
    \node[circle,fill,inner sep=1pt] (a) at (1,0) {};
    \node[circle,fill,inner sep=1pt] (b) at (2,0) {};
    \node[circle,fill,inner sep=1pt] (c) at (3,0) {};
    \node[circle,fill,inner sep=1pt] (f) at (3,-0.5) {};
    
    \draw (a) -- (b);
    \draw (b) -- (c);
    \draw (c) -- ++(0.5,0);
    \draw (c) -- (f);

    \node at (4,0) {$\ldots$};
    \draw (4.5,0) -- ++(0.5,0);

    \node[circle,fill,inner sep=1pt] (d) at (5,0) {};
    \node[circle,fill,inner sep=1pt] (e) at (6,0) {};
    \draw (d) -- (e);
   \end{tikzpicture}
  \end{subfigure}
  \caption{Simply-laced Dynkin diagrams of types $A_n,D_n$ and $E_n$.}
  \label{fig:dynkin-diagrams}
 \end{figure}

 Concretely, let $k$ be a field, let $\Lambda \coloneqq kQ$ denote the path
 algebra of $Q$ and $\mathcal{D} \coloneqq D^{b}(\mod \Lambda)$ the bounded
 derived category of finitely generated $\Lambda$-modules with shift functor
 $[1]$. Let $G:\mathcal{D} \to \mathcal{D}$ be a triangle functor with some
 necessary additional properties (see~\cite{k}). We denote by $\mathcal{D} / G$
 the corresponding orbit category, that is, a category with objects the
 $G$-orbits of objects of $\mathcal{D}$ and with morphisms
 \[
  \hom_{\mathcal{D} / G}(\bar{X},\bar{Y}) = \coprod_{i \in \Z}
  \hom_{\mathcal{D}}(G^{i}X,Y)
 \]
 where $X,Y \in \mathcal{D}$ and $\bar{X},\bar{Y}$ are the corresponding objects
 of $\mathcal{D} / G$. In~\cite{k}, it is shown that $\mathcal{D} / G$ is
 triangulated, the shift functor in $\mathcal{D} / G$ is induced by $[1]$ -- we
 shall denote it the same -- and that $\mathcal{D} / G$ is Krull-Schmidt. We
 write $\ext^{1}(U,V) \coloneqq \hom(U,V[1])$ where the morphisms are considered
 in $\mathcal{D}$ or $\mathcal{D} / G$.

 Henceforth, we focus on $\mathcal{D} / G$ with the special choice $G \coloneqq
 \tau^{-1}[1]$ where $\tau$ is the AR translation in $\mathcal{D}$ (induced by
 $D \mathrm{Tr}$ on non-projective indecomposable modules in $\mod \Lambda$ and
 satisfying $\tau(P) = I[-1]$ for $P$ indecomposable projective and $I$
 indecomposable injective with $\soc I \cong P / \rad P$). Because of its
 connection to the theory of cluster algebras, we call it the \emph{cluster
 category} associated to $\Lambda$. When $k$ is algebraically closed and $Q$ of
 Dynkin type, the cluster category $\mathcal{C} \coloneqq \mathcal{D} / G$
 depends (as shown in~\cite{bmrrt}) only on the underlying graph $\Delta$ of $Q$
 and the choice of orientation is immaterial. We write $\mathcal{C} =
 \mathcal{C}(\Delta)$. Most importantly, in this case, the combinatorics of the
 cluster algebra $\mathcal{A}(Q)$ can be fully recovered in terms of
 $\ext^{1}$-groups of $\mathcal{C}$. In particular, the clusters of
 $\mathcal{A}(Q)$ are in 1-1 correspondence with the cluster tilting objects of
 $\mathcal{C}$ -- objects $T \in \mathcal{C}$ such that $\ext^{1}(T,T) = 0$ and
 $T$ has the maximal number of non-isomorphic direct summands.

 In~\cite{fst}, cluster algebras of \emph{surface} type are introduced. These
 are cluster algebras whose underlying quiver arises from ideal triangulations
 of bordered surfaces with marked points -- connected oriented 2-dimensional
 Riemann surfaces with a finite number of distinguished points in their closure.
 Both cluster algebras of finite type and those of surface type share a
 `2-dimensional quality'. In the case of cluster algebras of surface type, it is
 the dimension of the surface; the 2-dimensionality of cluster algebras of
 finite type is reflected in the fact that the associated cluster category is
 2-Calabi-Yau (see~\cite[Proposition~1.7]{bmrrt}). It seems natural to seek
 higher-dimensional analogues of these constructions.

 The only cluster algebras of both finite and surface type are those with
 underlying quiver of Dynkin type $A$ or $D$. Quivers of types not Dynkin fail
 to produce cluster algebras of finite type and quivers of Dynkin type $E$ fail
 to correspond to any triangulation of any bordered surface with marked points.
 As the combinatorial properties of $\mathcal{A}(Q)$ depend only on the
 underlying graph of $Q$, the main focus shifts to the path algebra $kQ$ (with
 $Q$ being arbitrarily oriented) and the associated cluster category.

 The paper~\cite{ot} discusses the connection between $d$-Auslander algebras of
 $kA_n$ (with $A_n$ meaning the linearly-oriented quiver of Dynkin type $A$ on
 $n$ vertices) and cyclic polytopes of dimension $2d$ on $n + 2d$ vertices. 

 The \emph{cyclic polytope} $C(m,\delta)$ is the convex hull of $m$ distinct
 points on the moment curve $t \mapsto (t,t^2,\ldots,t ^{\delta})$ of dimension
 $\delta$. It is the most natural higher-dimensional analogue of a disc with a
 finite number of distinguished points on its boundary whose triangulations give
 rise to quivers of Dynkin type $A$ (see~\cite{fst}).

 The notions of \emph{$n$-cluster tilting} modules and
 \emph{$n$-representation-finite} algebras are introduced in~\cite{iya}
 and~\cite{io}, respectively. Suppose $\Lambda$ is any finite-dimensional
 algebra over a field $k$. A module $M \in \mod \Lambda$ is called $n$-cluster
 tilting if
 \begin{align*}
  \add M &= \{X \in \mod \Lambda \mid \ext^{i}(X,M) = 0 \; \forall i \in
  \{1,\ldots,n-1\}\}\\
         &= \{X \in \mod\Lambda \mid \ext^{i}(M,X) = 0 \; \forall i \in
         \{1,\ldots,n-1\}\}.
 \end{align*}
 If $\Lambda$ has an $n$-cluster tilting module and also $\gldim\Lambda \leq n$,
 then $\Lambda$ is called $n$-representation-finite. In such case, the
 endomorphism algebra $\Gamma \coloneqq \End_{\Lambda}(M)$ is called the
 $n$-Auslander algebra of $\Lambda$ and denoted $\Lambda^{(n)}$. Iyama
 (\cite{iya}) shows that if $\Lambda$ is $n$-representation-finite, then
 $\End_{\Lambda}(M)$ is $(n+1)$-representation-finite, allowing for an inductive
 construction of higher Auslander algebras of a $1$-representation-finite (that
 is, representation-finite and hereditary) algebra. Since representation-finite
 hereditary algebras are precisely the path algebras of Dynkin quivers, the list
 of algebras to study narrows dramatically. Moreover, for such algebras, Iyama
 gives a complete description of their higher Auslander algebras in terms of
 their AR quivers and relations (see~\cite[Section~6]{iya}).

 In~\cite{ot}, Oppermann and Thomas focus chiefly on the path algebra $\Lambda
 \coloneqq kA_n$, its $(d-1)$-st higher Auslander algebras (denoted $A^{d}_n$)
 and the associated (generalized) cluster category. They discover a bijection
 \[
  \left\{ 
   \parbox{.17\textwidth}{
    \centering
    triangulations of\\
    $C(n+2d,2d)$
   }
  \right\}
  \longleftrightarrow
  \left\{ 
   \parbox{.27\textwidth}{
    \centering
    basic tilting modules of $A^{d}_n$\\
    contained in $\add\prescript{}{A^{d}_n}{M}$
   }
  \right\},
 \]
 where $\prescript{}{A^{d}_n}{M}$ is the $d$-cluster tilting module of $A^{d}_n$
 which is unique up to multiplicity (\cite{iya}). Furthermore,
 in~\cite[Section~5]{ot}, the authors generalize the concept of a triangulated
 cluster category to a $(d+2)$-angulated cluster category for any
 $d$-represention-finite algebra $\Lambda$. See~\cite{gko} for a definition of
 an $n$-angulated category. Specifically, a full subcategory $\mathcal{A}$ of a
 triangulated category $\mathcal{B}$ is called $\delta$-cluster tilting if
 $\mathcal{A}$ is functorially finite in $\mathcal{B}$ (see~\cite{as}) and
 $\mathcal{A}$ coincides with both
 \[
  \{T \in \mathcal{B} \mid \hom_{\mathcal{B}}(\mathcal{A},T[i]) = 0 \; \forall
  i \in \{1,\ldots,\delta-1\}\} \quad \text{and} \quad \{T \in \mathcal{B} \mid
  \hom_{\mathcal{B}}(T,\mathcal{A}[i]) = 0 \; \forall i \in
  \{1,\ldots,\delta-1\}\}.
 \]
 If $\add T$ is a $\delta$-cluster tilting subcategory of $\mathcal{B}$, the
 object $T$ is called $\delta$-cluster tilting. If $\Lambda$ is
 $d$-representation-finite, $S$ denotes the Serre functor on $\mathcal{D} =
 D^{b}(\mod\Lambda)$ and $S_d$ its $d$-th desuspension $S[-d]$, then the
 category
 \[
  \mathcal{U} \coloneqq \add \{S^{i}_d\Lambda \mid i \in \Z\} = \add \{M[id]
  \mid i \in \Z\},
 \]
 where $M$ is a $d$-cluster tilting $\Lambda$-module, is a $d$-cluster tilting
 subcategory of $\mathcal{D}$ (see~\cite[Theorem 1.23]{iya}). The
 $(d+2)$-angulated cluster category of $\Lambda$ is then defined as the orbit
 category
 \[
  \mathcal{O}_{\Lambda} \coloneqq \mathcal{U} / S_{2d}.
 \]
 It is shown in~\cite[Theorem~5.2]{ot} that this category is Krull-Schmidt and
 $2d$-Calabi-Yau and the isomorphism classes of indecomposable objects of
 $\mathcal{O}_{\Lambda}$ are in 1-1 correspondence with the indecomposable
 summands of $M \oplus \Lambda[d]$. Finally, in the case of $\Lambda = kA_n$,
 the authors prove there is a correspondence
 \[
  \left\{ 
   \parbox{.19\textwidth}{
    \centering
    internal $d$-simplices\\
    of $C(n+2d+1,2d)$
   }
  \right\}
  \longleftrightarrow
  \left\{ 
   \parbox{.16\textwidth}{
    \centering
    indecomposable
    objects of $\mathcal{O}_{A^{d}_n}$
   }
  \right\},
 \]
 which induces the correspondence
 \[
  \left\{ 
   \parbox{.17\textwidth}{
    \centering
    triangulations of\\
    $C(n+2d+1,2d)$
   }
  \right\}
  \longleftrightarrow
  \left\{ 
   \parbox{.19\textwidth}{
    \centering
    basic cluster tilting
    objects of $\mathcal{O}_{A^{d}_n}$
   }
  \right\}.
 \]

 In recent strongly related papers~\cite{djw} and~\cite{djy}, an alternative
 combinatorial description of (perfect derived categories of) higher Auslander
 algebras of type $A$ is given from the viewpoint of symplectic geometry and
 simplicial theory, respectively.

 It is my aim to reproduce these results (namely those in~\cite{ot}) for the
 path algebras of quivers of Dynkin type $D$ and $E$ and their associated
 cluster categories. In the case of type $D$, such quivers correspond to ideal
 triangulations of a disc with finite number of marked points on its boundary
 and a single marked point in its interior (\cite{fst}). The obvious
 higher-dimensional analogue is again the cyclic polytope with one distinguished
 point in its interior. Unfortunately, quivers of Dynkin type $E$ do not arise
 from triangulations of bordered surfaces with marked points and thus one is
 required to search for a combinatorial counterpart which is not necessarily
 `geometric'.

 It should be noted that the 2-dimensional (or classical) cluster categories of
 type $D$ were already studied from a geometric point of view in~\cite{schif}.

 Next, I intend to focus on higher-dimensional analogues and the cluster
 categories of path algebras $kQ$ which are not represen\-tation-finite and
 hereditary, that is, $Q$ is not Dynkin. Since full combinatorial description
 doesn't seem at all plausible in the current state of research, I hope to find
 hints towards such a goal by studying path algebras of quivers which are `close
 enough' to Dynkin quivers so that the theory doesn't get completely out of
 hand. Concretely, I shall focus on quivers whose underlying graphs are
 so-called \emph{affine} Dynkin diagrams
 (figure~\ref{fig:affine-dynkin-diagrams}).
 \begin{figure}[ht]
  \centering
  \begin{subfigure}[b]{.41\textwidth}
   \begin{tikzpicture}
    \node at (0,0) {$\tilde{A}_n$};
    \node[circle,fill,inner sep=1pt] (a) at (1,0) {};
    \node[circle,fill,inner sep=1pt] (b) at (2,0) {};
    \node[circle,fill,inner sep=1pt] (c) at (3,0) {};
    
    \draw (a) -- (b);
    \draw (b) -- (c);
    \draw (c) -- ++(0.5,0);

    \node at (4,0) {$\ldots$};
    \draw (4.5,0) -- ++(0.5,0);

    \node[circle,fill,inner sep=1pt] (d) at (5,0) {};
    \node[circle,fill,inner sep=1pt] (e) at (6,0) {};
    \draw (d) -- (e);

    \node[circle,fill,inner sep=1pt] (f) at (3.5,-1) {};
    \draw (a) -- (f);
    \draw (e) -- (f);
   \end{tikzpicture}
   \vspace{1em}
  \end{subfigure}\linebreak
  \begin{subfigure}[b]{.41\textwidth}
   \begin{tikzpicture}
    \node at (0,0) {$\tilde{D}_n$};
    \node[circle,fill,inner sep=1pt] (a1) at (1,0.5) {};
    \node[circle,fill,inner sep=1pt] (a2) at (1,-0.5) {};
    \node[circle,fill,inner sep=1pt] (b) at (2,0) {};
    \node[circle,fill,inner sep=1pt] (c) at (3,0) {};
    
    \draw (a1) -- (b);
    \draw (a2) -- (b);
    \draw (b) -- (c);
    \draw (c) -- ++(0.5,0);

    \node at (4,0) {$\ldots$};
    \draw (4.5,0) -- ++(0.5,0);

    \node[circle,fill,inner sep=1pt] (d) at (5,0) {};
    \node[circle,fill,inner sep=1pt] (e1) at (6,0.5) {};
    \node[circle,fill,inner sep=1pt] (e2) at (6,-0.5) {};
    \draw (d) -- (e1);
    \draw (d) -- (e2);
   \end{tikzpicture}
   \vspace{1em}
  \end{subfigure}\linebreak
  \begin{subfigure}[b]{.41\textwidth}
   \begin{tikzpicture}
    \node at (0,0) {$\tilde{E}_6$};
    \node[circle,fill,inner sep=1pt] (a) at (1,0) {};
    \node[circle,fill,inner sep=1pt] (b) at (2,0) {};
    \node[circle,fill,inner sep=1pt] (c) at (3,0) {};
    \node[circle,fill,inner sep=1pt] (f) at (3,-0.5) {};
    \node[circle,fill,inner sep=1pt] (g) at (3,-1) {};
    \node[circle,fill,inner sep=1pt] (d) at (4,0) {};
    \node[circle,fill,inner sep=1pt] (e) at (5,0) {};
    
    \draw (a) -- (b);
    \draw (b) -- (c);
    \draw (c) -- (d);
    \draw (d) -- (e);
    \draw (c) -- (f);
    \draw (f) -- (g);
   \end{tikzpicture}
  \end{subfigure}\linebreak
  \begin{subfigure}[b]{.41\textwidth}
   \begin{tikzpicture}
    \node at (-1,0) {$\tilde{E}_7$};
    \node[circle,fill,inner sep=1pt] (z) at (0,0) {};
    \node[circle,fill,inner sep=1pt] (a) at (1,0) {};
    \node[circle,fill,inner sep=1pt] (b) at (2,0) {};
    \node[circle,fill,inner sep=1pt] (c) at (3,0) {};
    \node[circle,fill,inner sep=1pt] (f) at (3,-0.5) {};
    \node[circle,fill,inner sep=1pt] (d) at (4,0) {};
    \node[circle,fill,inner sep=1pt] (e) at (5,0) {};
    \node[circle,fill,inner sep=1pt] (g) at (6,0) {};
    
    \draw (a) -- (b);
    \draw (b) -- (c);
    \draw (c) -- (d);
    \draw (d) -- (e);
    \draw (c) -- (f);
    \draw (z) -- (a);
    \draw (e) -- (g);
   \end{tikzpicture}
  \end{subfigure}\linebreak
  \begin{subfigure}[b]{.41\textwidth}
   \begin{tikzpicture}
    \node at (-1,0) {$\tilde{E}_7$};
    \node[circle,fill,inner sep=1pt] (z) at (0,0) {};
    \node[circle,fill,inner sep=1pt] (a) at (1,0) {};
    \node[circle,fill,inner sep=1pt] (b) at (2,0) {};
    \node[circle,fill,inner sep=1pt] (c) at (3,0) {};
    \node[circle,fill,inner sep=1pt] (f) at (2,-0.5) {};
    \node[circle,fill,inner sep=1pt] (d) at (4,0) {};
    \node[circle,fill,inner sep=1pt] (e) at (5,0) {};
    \node[circle,fill,inner sep=1pt] (g) at (6,0) {};
    \node[circle,fill,inner sep=1pt] (h) at (7,0) {};
    
    \draw (a) -- (b);
    \draw (b) -- (c);
    \draw (c) -- (d);
    \draw (d) -- (e);
    \draw (b) -- (f);
    \draw (z) -- (a);
    \draw (e) -- (g);
    \draw (g) -- (h);
   \end{tikzpicture}
  \end{subfigure}\linebreak
  \caption{Affine Dynkin diagrams of types $\tilde{A}_n$, $\tilde{D}_n$,
  $\tilde{E}_6$, $\tilde{E}_7$ and $\tilde{E}_8$.}
  \label{fig:affine-dynkin-diagrams}
 \end{figure}

 Several required generalizations of the aforewritten theory for
 $n$-representation-finite algebras have already been constructed.
 In~\cite{hio}, \emph{$n$-representation-infinite} algebras are defined in
 general and, particularly, $n$-representation-infinite algebras of type
 $\tilde{A}$ are constructed as certain graded orbit algebras. The authors then
 proceed to show that the degree 0 parts of these algebras are indeed the path
 algebras of quivers of type $\tilde{A}$. Whether such construction proves
 significant in the study of combinatorial properties of these algebras remains
 to be seen. Moreover, by~\cite[Theorem 4.18]{hio}, the category of finitely
 generated modules over any $n$-representation-infinite algebra splits into its
 $n$-preprojective, $n$-regular and $n$-preinjective components. This fact has
 been successfully used to, for example, study the combinatorics of AR quivers
 of $n$-representation-infinite path algebras $kQ$ through the truncations of
 the (generalized) \emph{repetition quiver} $\Z Q$ -- a quiver whose vertices
 are elements of $\Z \times Q_0$ and for each arrow $i \to j$ in $Q_1$ there are
 arrows $(p,i) \to (p,j)$ and $(p-1,j) \to (p,i)$ where $p$ ranges over $\Z$
 (see~\cite{gll}).

 To the best of my knowledge, there has thus far not been developed a notion of
 a generalized $n$-Auslander algebra for algebras where the existence of an
 $n$-cluster tilting module is not guaranteed which would allow for an iterative
 construction similar in nature to the one for algebras
 $n$-representation-finite. I do not expect to be able to develop such theory in
 full generality, but I expect the path algebras of affine Dynkin quivers will
 provide a more hospitable place to such endeavour.

 There exist multiple definitions of `higher-dimensional' cluster categories.
 One is due to C. Amiot (\cite{amiot2} and~\cite{amiot1}) and stays true to the
 original definition in~\cite{k}. She defines the triangulated category
 $\mathcal{C}^{\delta}_{\Lambda}$ of any algebra $\Lambda$ with $\gldim\Lambda
 \leq \delta$ as the \emph{triangulated hull} of the orbit category
 $D^{b}(\mod\Lambda) / S_{\delta}$ where, again, $S$ is the Serre functor on
 $D^{b}(\mod\Lambda)$ and $S_{\delta} \coloneqq S[-\delta]$. She then proceeds
 to prove that as long as $\mathcal{C}^{\delta}_{\Lambda}$ is $\hom$-finite, it
 is $\delta$-Calabi-Yau and admits a $\delta$-cluster tilting object. The
 $(d+2)$-angulated cluster category $\mathcal{O}_{\Lambda}$ for $\Lambda$
 $n$-representation-finite defined in~\cite{ot} and discussed above is a
 $d$-cluster tilting subcategory of $\mathcal{C}^{2d}_{\Lambda}$.

 Another approach is due to Guo (\cite{guo}), generalizing the construction from~\cite[Section~4]{amiot1}. Given a differential-graded
 $k$-algebra $\Lambda$, we denote by $\per \Lambda$ its \emph{perfect derived
 category}, that is, the smallest triangulated subcategory of $D(\Lambda)$ --
 the \emph{unbounded} derived category of $\Lambda$ -- containing $\Lambda$
 which is closed under forming direct summands. Supposing $\Lambda$ satisfies
 certain additional properties (laid out in~\cite[Section~3.2]{guo}), in
 particular, assuming $\Lambda$ is $(m+2)$-Calabi-Yau as a bimodule for an
 integer $m$, the \emph{generalized $m$-cluster category} associated with
 $\Lambda$ is defined as the triangulated quotient $\per \Lambda /
 D^{b}(\Lambda)$. It is $(m+1)$-Calabi-Yau and hosts an $m$-cluster tilting
 object. This generalized notion of a cluster category can be related to path
 algebras of finite quivers by works of Ginzburg and Keller. For a finite quiver
 $Q$ with potential $W$, Ginzburg defines (\cite[Section~4.2]{ginz}) certain
 differential-graded $k$-algebra $\Gamma(Q,W)$ whose underlying graded algebra
 is the path algebra $k \hat{Q}$ with $\hat{Q}$ being a graded quiver
 constructed from $Q$. The algebra $\Gamma(Q,W)$ is later shown by Keller
 (\cite{k2}) to satisfy the criteria laid out in~\cite[Section~3.2]{guo}. In
 particular, this algebra is always $3$-Calabi-Yau, suggesting possibilities of
 constructions of this nature for higher-dimensional analogues of the path
 algebra $kQ$.

 No parallels between $n$-representation-infinite algebras or generalized
 cluster categories and purely combinatorial structures have thus far been
 drawn. While focusing on the path algebras of affine Dynkin quivers, it seems
 necessary to generalize the construction of higher Auslander algebras
 from~\cite{iya} to account for the loss of a guaranteed existence of an
 $n$-cluster tilting module. Similarly, an alternative construction of the
 generalized cluster categories associated to such path algebras might be
 required to remedy the fact that the orientation of the quiver has now
 influence on the overall structure. I hope to prove that, at least in this
 `almost-Dynkin' case, there still exist natural combinatorial descriptions of
 the tilting modules of these path algebras and the tilting objects of their
 associated generalized cluster categories which are mutually compatible.

 \printbibliography
\end{document}
