\documentclass[a4paper,oneside,svgnames]{amsart}

\usepackage[T1]{fontenc}
\usepackage[utf8]{inputenc}

\usepackage[margin=1in]{geometry}
\usepackage{xcolor}
\usepackage[colorlinks=true,citecolor=Green,urlcolor=Black]{hyperref}
\usepackage[backend=biber,style=alphabetic,sorting=ynt]{biblatex}
\addbibresource{refs.bib}
\usepackage{tikz}
\usetikzlibrary{arrows,math}
\usepackage[noabbrev]{cleveref}
\usepackage{caption}
\usepackage{subcaption}
\usepackage{mathtools}
\renewcommand{\thesubsection}{\arabic{subsection}}

% Theorems
\theoremstyle{plain}
\newtheorem{theorem}{Theorem}[section]
\newtheorem{proposition}[theorem]{Proposition}
\newtheorem{lemma}[theorem]{Lemma}

\theoremstyle{definition}
\newtheorem{definition}[theorem]{Definition}

\let\hom\relax
\let\mod\relax
\DeclareMathOperator{\ext}{Ext}
\DeclareMathOperator{\add}{add}
\DeclareMathOperator{\hom}{Hom}
\DeclareMathOperator{\End}{End}
\DeclareMathOperator{\mod}{mod}
\DeclareMathOperator{\soc}{soc}
\DeclareMathOperator{\rad}{rad}
\DeclareMathOperator{\per}{per}
\DeclareMathOperator{\gldim}{gl.dim}

\newcommand{\C}{\mathbb{C}}
\newcommand{\Z}{\mathbb{Z}}

\title{Combinatorial Properties of Higher Cluster Categories}
\author{Adam Klepáč}

\begin{document}
 \maketitle
 \section*{Summary}
 \setcounter{section}{1}

 The main objective of the project is to understand and describe the
 combinatorics of higher-dimension\-al analogues of the path algebras (in the
 sense of higher representation theory) of affine Dynkin quivers and their
 associated higher cluster categories, possibly and hopefully aiding in an
 eventual broader comprehension of the combinatorial structure of
 $n$-representation-infinite algebras.

 The first year will be dedicated to the study of the combinatorics of tilting
 modules in higher Auslander algebras of the path algebras of type $D$ and $E$
 Dynkin quivers and the cluster tilting objects of their associated higher 
 cluster categories. A full combinatorial description is already in place for
 these algebras and categories of Dynkin type $A$ by means of triangulations of
 cyclic polytopes. It remains to extend this theory to also include types $D$
 and $E$.

 In the next two years, I aim to generalize this theory further by studying path
 algebras of affine Dynkin quivers (of types $\tilde{A}$, $\tilde{D}$ and
 $\tilde{E}$) which are no longer representation-finite and thus the concept of
 higher Auslander algebras is not defined in this case. Due to the apparent
 proximity of these path algebras to those of pure Dynkin type, it seems
 reasonable to expect that a generalized version of a higher Auslander algebra
 can be constructed. For instance, there already exists a construction of
 $n$-representation-infinite algebras of type $\tilde{A}$. The search for a
 combinatorial description of the tilting modules of these algebras will mainly
 involve computing their Auslander-Reiten quivers and finding a regular
 criterion which can be used to check whether a given module is tilting. I
 intend to first find such a combinatorial model in the low-dimensional setting
 and see if it propagates well to higher dimensions. If it does not, it likely
 means the model is wrong, but it can also point to errors in my conception of
 the generalized higher Auslander algebras. Finally, for these algebras, I hope
 to find a suitable construction of a cluster category whose cluster tilting
 objects will share the same combinatorial model as the tilting modules of its
 associated algebra and, last but not least, its cluster tilting objects will in
 the low-dimensional case also correspond to the clusters of the connected
 cluster algebra. Higher cluster categories have already been defined and
 studied. I suspect the right construction will turn out to be a certain
 distinguished $d$-cluster tilting subcategory of the higher cluster category
 associated to the algebra in question just as it is the case for higher
 Auslander algebras of Dynkin type $A$.

\end{document}
