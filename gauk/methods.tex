\documentclass[a4paper,oneside,svgnames]{amsart}

\usepackage[T1]{fontenc}
\usepackage[utf8]{inputenc}

\usepackage[margin=1in]{geometry}
\usepackage{xcolor}
\usepackage[colorlinks=true,citecolor=Green,urlcolor=Black]{hyperref}
\usepackage[backend=biber,style=alphabetic,sorting=ynt]{biblatex}
\addbibresource{refs.bib}
\usepackage{tikz}
\usetikzlibrary{arrows,math}
\usepackage[noabbrev]{cleveref}
\usepackage{caption}
\usepackage{subcaption}
\usepackage{mathtools}
\renewcommand{\thesubsection}{\arabic{subsection}}

% Theorems
\theoremstyle{plain}
\newtheorem{theorem}{Theorem}[section]
\newtheorem{proposition}[theorem]{Proposition}
\newtheorem{lemma}[theorem]{Lemma}

\theoremstyle{definition}
\newtheorem{definition}[theorem]{Definition}

\let\hom\relax
\let\mod\relax
\DeclareMathOperator{\ext}{Ext}
\DeclareMathOperator{\add}{add}
\DeclareMathOperator{\hom}{Hom}
\DeclareMathOperator{\End}{End}
\DeclareMathOperator{\mod}{mod}
\DeclareMathOperator{\soc}{soc}
\DeclareMathOperator{\rad}{rad}
\DeclareMathOperator{\per}{per}
\DeclareMathOperator{\gldim}{gl.dim}

\newcommand{\C}{\mathbb{C}}
\newcommand{\Z}{\mathbb{Z}}

\title{Combinatorial Properties of Higher Cluster Categories}
\author{Adam Klepáč}

\begin{document}
 \maketitle
 \section*{Methods of Research}
 \setcounter{section}{1}

 The majority of the first year will be spent on research objective 1 -- the
 combinatorial description of higher Auslander algebras of path algebras of
 Dynkin type $D$ and $E$ and their associated cluster categories -- followed by
 a thorough study of the theory of generalized cluster categories (mainly
 \cite[Sections 3 and 4]{amiot1}, \cite[Chapters 5,6 and 7]{amiot2} and
 \cite[Chapters 2 and 3]{guo}), of $n$-representation-infinite algebras
 (\cite[Sections 1-5]{hio}) and $n$-angulated categories (the entirety
 of~\cite{gko}).

 The correspondence between quivers of Dynkin type $D$ and triangulations of
 discs with marked points on their boundaries and one in their interior is bound
 to prove crucial for this endeavour. Unlike in the case of triangulations of
 discs with marked points lying only on their boundaries (giving rise to quivers
 of type $A$), the existence of an inner marked point naturally leads to the
 concept of a self-folded triangle (a triangle with only two distinct sides).
 The first non-trivial task will entail generalizing this concept to higher
 dimensions and understanding the triangulations of a cyclic polytope with a
 distinguished point in its interior.

 In~\cite[Section 3]{ot}, an orientation of $A_n$ is chosen that guarantees the
 equality $\hom_{A^{1}_n}(P_i,P_j) = 0$ if and only if $i > j$ (here $P_i$
 denotes the projective cover of the simple module $S_i$ concentrated in vertex
 $i$). With the right approach, this choice makes the study of partial tilting
 modules of $A^{d}_n$ contained in $\add \prescript{}{A^{d}_n}{M}$ (here
 $\prescript{}{A^{d}_n}{M}$ denotes the cluster tilting module of $A^{d}_n$) --
 and also the indecomposable objects of the associated cluster category --
 smooth and elegant. Such an orientation cannot be chosen for the quivers of
 type $D$ or $E$ and thus irregularities will inevitably occur and need to be
 dealt with. More irregularities lurk within the connection between bistellar
 flips of the triangulations of cyclic polytopes and mutations of basic tilting
 modules of $A^{d}_n$ -- as well as mutations of the cluster tilting objects of
 the associated cluster category -- detailed in~\cite[Sections 4 and 6]{ot}. The
 issue lies in the fact that the folded edges in self-folded triangles cannot be
 flipped. This situation does not arise in the study of triangulations of discs
 without internal marked points and is likely to propagate to higher dimensions.
 It is already a predictor of irregular behaviour even in the low-dimensional
 case. The flip-mutation correspondence would imply that the indecomposable
 direct summand (of the tilting module -- or the cluster tilting object --
 determined by the given triangulation) corresponding to an unflippable edge (or
 facet in higher dimensions) cannot be substituted for another. Nonetheless,
 this statement is false for the (classical) cluster categories of acyclic
 quivers where any indecomposable summand of a cluster tilting object can be
 substituted. This points to the possibility that the correspondence between
 triangulations and tilting modules (or cluster tilting objects) will not be
 entirely bijective or the triangulations will have to be considered up to a
 certain kind of equivalence.

 Finally, as quivers of type $E$ do not arise from triangulations of bordered
 surfaces with marked points, a combinatorial structure of not necessarily
 geometric nature needs to be found to deal with this case. Hopefully, the same
 structure can be used for all three such quivers.

 The next two years will be dedicated to the generalization of the theory of
 higher Auslander algebras of representation-finite hereditary algebras to the
 path algebras of affine Dynkin quivers, the study of their associated
 generalized cluster categories and the combinatorial properties of both.

 First, I need to conceive a suitable notion of a generalized higher Auslander
 algebra, that is, work on research objective 2. The construction
 in~\cite[Section~5]{hio} of the $n$-representation-infinite algebras of type
 $\tilde{A}$ can easily be generalized to types $\tilde{D}$ or $\tilde{E}$ by
 simply considering root systems of type $D_n$ or $E_n$ at its very beginning.
 In the case of type $\tilde{A}$, these graded algebras are bimodule
 $(n+1)$-Calabi-Yau and for $n = 1$ their degree 0 part are precisely the path
 algebras of quivers of type $\tilde{A}$ for non-cyclic orientations. It remains
 to prove that this is also the case if the initial root system is different.
 However, it may well happen that a complete understanding of the combinatorics
 of their tilting modules proves unrealistic or exceedingly difficult. In such a
 case, I will strive to reach a simpler construction which exploits the apparent
 proximity of the path algebras of quivers of (pure) Dynkin type to those of
 type affine Dynkin. This would allow me to successfully put to use the
 knowledge gained from~\cite{ot} and research objective 1 at the cost of
 generality.

 The next step is to understand the combinatorics of tilting modules in the
 algebras from the previous paragraph. The most natural place to start are the
 Auslander-Reiten quivers of the path algebras of affine Dynkin quivers which
 have already been described and studied. I trust a reasonable approach is to
 find a combinatorial counterpart in this low-dimensional case before moving on
 to higher dimensions. The reason for that is the following: a combinatorial
 model for the tilting modules of path algebras of affine Dynkin quivers may
 serve as a sort of `benchmark' for the naturality of the definition of their
 generalized higher Auslander algebras. Should the model be too simple or just
 way off the mark in higher dimensions, it will call into question not only the
 correctness of the model itself but also the method wherewith said algebras
 were conceived.

 This step also naturally includes the computation of AR quivers of the
 generalized higher Auslander algebras. Should it prove hard to achieve, I
 intend to first study the AR quivers of their $n$-preprojective,
 $n$-preinjective and $n$-regular subcategories which form disjoint parts of the
 AR quiver of the entire $n$-representation-infinite algebra in question
 (see~\cite[Theorem 4.18]{hio}).

 Research objective 4 is quite likely the most difficult to accomplish. In order
 to define a suitable generalized cluster category associated to an
 $n$-representation-infinite algebra $\Lambda$ defined in objective 2, I shall
 first try to mimic (with necessary adjustments) the construction
 from~\cite[Section 5]{ot} of a $(d+2)$-angulated $d$-cluster tilting
 subcategory of $\mathcal{C}^{2d}_{\Lambda}$ (in the sense of~\cite{amiot1}). It
 would be a remarkable display of symmetry should their cluster tilting objects
 be modelled by the structure discovered in objective 3. The pessimist in me is
 inclined to think this will not be the case and a different approach will have
 to be taken, quite possibly utilizing the further generalizations of the theory
 of higher cluster categories depicted in~\cite{guo}.

\end{document}
